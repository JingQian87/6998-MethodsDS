\documentclass[]{article}
\usepackage{lmodern}
\usepackage{amssymb,amsmath}
\usepackage{ifxetex,ifluatex}
\usepackage{fixltx2e} % provides \textsubscript
\ifnum 0\ifxetex 1\fi\ifluatex 1\fi=0 % if pdftex
  \usepackage[T1]{fontenc}
  \usepackage[utf8]{inputenc}
\else % if luatex or xelatex
  \ifxetex
    \usepackage{mathspec}
  \else
    \usepackage{fontspec}
  \fi
  \defaultfontfeatures{Ligatures=TeX,Scale=MatchLowercase}
\fi
% use upquote if available, for straight quotes in verbatim environments
\IfFileExists{upquote.sty}{\usepackage{upquote}}{}
% use microtype if available
\IfFileExists{microtype.sty}{%
\usepackage{microtype}
\UseMicrotypeSet[protrusion]{basicmath} % disable protrusion for tt fonts
}{}
\usepackage[margin=1in]{geometry}
\usepackage{hyperref}
\hypersetup{unicode=true,
            pdftitle={Empirical Methods of Data Science},
            pdfauthor={Amal Alabdulkarim (aa4235), Jing Qian (jq2282)},
            pdfborder={0 0 0},
            breaklinks=true}
\urlstyle{same}  % don't use monospace font for urls
\usepackage{color}
\usepackage{fancyvrb}
\newcommand{\VerbBar}{|}
\newcommand{\VERB}{\Verb[commandchars=\\\{\}]}
\DefineVerbatimEnvironment{Highlighting}{Verbatim}{commandchars=\\\{\}}
% Add ',fontsize=\small' for more characters per line
\usepackage{framed}
\definecolor{shadecolor}{RGB}{248,248,248}
\newenvironment{Shaded}{\begin{snugshade}}{\end{snugshade}}
\newcommand{\AlertTok}[1]{\textcolor[rgb]{0.94,0.16,0.16}{#1}}
\newcommand{\AnnotationTok}[1]{\textcolor[rgb]{0.56,0.35,0.01}{\textbf{\textit{#1}}}}
\newcommand{\AttributeTok}[1]{\textcolor[rgb]{0.77,0.63,0.00}{#1}}
\newcommand{\BaseNTok}[1]{\textcolor[rgb]{0.00,0.00,0.81}{#1}}
\newcommand{\BuiltInTok}[1]{#1}
\newcommand{\CharTok}[1]{\textcolor[rgb]{0.31,0.60,0.02}{#1}}
\newcommand{\CommentTok}[1]{\textcolor[rgb]{0.56,0.35,0.01}{\textit{#1}}}
\newcommand{\CommentVarTok}[1]{\textcolor[rgb]{0.56,0.35,0.01}{\textbf{\textit{#1}}}}
\newcommand{\ConstantTok}[1]{\textcolor[rgb]{0.00,0.00,0.00}{#1}}
\newcommand{\ControlFlowTok}[1]{\textcolor[rgb]{0.13,0.29,0.53}{\textbf{#1}}}
\newcommand{\DataTypeTok}[1]{\textcolor[rgb]{0.13,0.29,0.53}{#1}}
\newcommand{\DecValTok}[1]{\textcolor[rgb]{0.00,0.00,0.81}{#1}}
\newcommand{\DocumentationTok}[1]{\textcolor[rgb]{0.56,0.35,0.01}{\textbf{\textit{#1}}}}
\newcommand{\ErrorTok}[1]{\textcolor[rgb]{0.64,0.00,0.00}{\textbf{#1}}}
\newcommand{\ExtensionTok}[1]{#1}
\newcommand{\FloatTok}[1]{\textcolor[rgb]{0.00,0.00,0.81}{#1}}
\newcommand{\FunctionTok}[1]{\textcolor[rgb]{0.00,0.00,0.00}{#1}}
\newcommand{\ImportTok}[1]{#1}
\newcommand{\InformationTok}[1]{\textcolor[rgb]{0.56,0.35,0.01}{\textbf{\textit{#1}}}}
\newcommand{\KeywordTok}[1]{\textcolor[rgb]{0.13,0.29,0.53}{\textbf{#1}}}
\newcommand{\NormalTok}[1]{#1}
\newcommand{\OperatorTok}[1]{\textcolor[rgb]{0.81,0.36,0.00}{\textbf{#1}}}
\newcommand{\OtherTok}[1]{\textcolor[rgb]{0.56,0.35,0.01}{#1}}
\newcommand{\PreprocessorTok}[1]{\textcolor[rgb]{0.56,0.35,0.01}{\textit{#1}}}
\newcommand{\RegionMarkerTok}[1]{#1}
\newcommand{\SpecialCharTok}[1]{\textcolor[rgb]{0.00,0.00,0.00}{#1}}
\newcommand{\SpecialStringTok}[1]{\textcolor[rgb]{0.31,0.60,0.02}{#1}}
\newcommand{\StringTok}[1]{\textcolor[rgb]{0.31,0.60,0.02}{#1}}
\newcommand{\VariableTok}[1]{\textcolor[rgb]{0.00,0.00,0.00}{#1}}
\newcommand{\VerbatimStringTok}[1]{\textcolor[rgb]{0.31,0.60,0.02}{#1}}
\newcommand{\WarningTok}[1]{\textcolor[rgb]{0.56,0.35,0.01}{\textbf{\textit{#1}}}}
\usepackage{graphicx,grffile}
\makeatletter
\def\maxwidth{\ifdim\Gin@nat@width>\linewidth\linewidth\else\Gin@nat@width\fi}
\def\maxheight{\ifdim\Gin@nat@height>\textheight\textheight\else\Gin@nat@height\fi}
\makeatother
% Scale images if necessary, so that they will not overflow the page
% margins by default, and it is still possible to overwrite the defaults
% using explicit options in \includegraphics[width, height, ...]{}
\setkeys{Gin}{width=\maxwidth,height=\maxheight,keepaspectratio}
\IfFileExists{parskip.sty}{%
\usepackage{parskip}
}{% else
\setlength{\parindent}{0pt}
\setlength{\parskip}{6pt plus 2pt minus 1pt}
}
\setlength{\emergencystretch}{3em}  % prevent overfull lines
\providecommand{\tightlist}{%
  \setlength{\itemsep}{0pt}\setlength{\parskip}{0pt}}
\setcounter{secnumdepth}{0}
% Redefines (sub)paragraphs to behave more like sections
\ifx\paragraph\undefined\else
\let\oldparagraph\paragraph
\renewcommand{\paragraph}[1]{\oldparagraph{#1}\mbox{}}
\fi
\ifx\subparagraph\undefined\else
\let\oldsubparagraph\subparagraph
\renewcommand{\subparagraph}[1]{\oldsubparagraph{#1}\mbox{}}
\fi

%%% Use protect on footnotes to avoid problems with footnotes in titles
\let\rmarkdownfootnote\footnote%
\def\footnote{\protect\rmarkdownfootnote}

%%% Change title format to be more compact
\usepackage{titling}

% Create subtitle command for use in maketitle
\newcommand{\subtitle}[1]{
  \posttitle{
    \begin{center}\large#1\end{center}
    }
}

\setlength{\droptitle}{-2em}

  \title{Empirical Methods of Data Science}
    \pretitle{\vspace{\droptitle}\centering\huge}
  \posttitle{\par}
  \subtitle{Assignment: Data analysis of infant mortality rates}
  \author{Amal Alabdulkarim (aa4235), Jing Qian (jq2282)}
    \preauthor{\centering\large\emph}
  \postauthor{\par}
    \date{}
    \predate{}\postdate{}
  

\begin{document}
\maketitle

\hypertarget{april-10.-new-findings}{%
\subsection{April 10. New findings}\label{april-10.-new-findings}}

在各种年龄的婴儿的死亡率之后,应该算下这些之间的correlation
(或者之后)。这样就说明了之后对于infant的结果(race,
year)可以适用于中间的细节~

\hypertarget{step-1.-load-data}{%
\subsection{Step 1. Load data}\label{step-1.-load-data}}

\begin{Shaded}
\begin{Highlighting}[]
\KeywordTok{library}\NormalTok{(}\StringTok{'readxl'}\NormalTok{)}
\NormalTok{data =}\StringTok{ }\KeywordTok{read_excel}\NormalTok{(}\StringTok{"table011.xlsx"}\NormalTok{)}
\end{Highlighting}
\end{Shaded}

\begin{verbatim}
## Warning in strptime(x, format, tz = tz): unknown timezone 'zone/tz/2018i.
## 1.0/zoneinfo/America/New_York'
\end{verbatim}

\begin{Shaded}
\begin{Highlighting}[]
\KeywordTok{names}\NormalTok{(data)[}\DecValTok{2}\NormalTok{]<-}\StringTok{"Year"}
\KeywordTok{View}\NormalTok{(data)}
\end{Highlighting}
\end{Shaded}

In this table, infant mortality rates are shown in different races and
years. Five levels of races are discussed: all races, race of child is
white, race of mother is white, race of child is black or African
American and race of mother is black or African American. The time span
of the data is from 1950 to 2016. The mortality rates are shown in four
catetories: infant, neonatal under 28 days, neonatol under 7 days and
postneonatal. \footnote{Infant (aged \textless{} 1 year), neonatal (aged
  \textless{} 28 days), postneonatal (aged 28 days to 11 months). So the
  infant death number = the neonatal death number + the postneonatal
  death.}

\hypertarget{step-2-data-exporation}{%
\subsection{Step 2: Data exporation}\label{step-2-data-exporation}}

\hypertarget{difference-in-mortality-rate-due-to-the-age-of-infants}{%
\subsubsection{2.1. Difference in Mortality rate due to the age of
infants}\label{difference-in-mortality-rate-due-to-the-age-of-infants}}

First, let's look at the mortality rate of infants in different age.

\begin{Shaded}
\begin{Highlighting}[]
\KeywordTok{library}\NormalTok{(}\StringTok{"ggplot2"}\NormalTok{)}
\end{Highlighting}
\end{Shaded}

\begin{verbatim}
## Warning: replacing previous import by 'rlang::dots_n' when loading 'dplyr'
\end{verbatim}

\begin{Shaded}
\begin{Highlighting}[]
\NormalTok{a =}\StringTok{ }\KeywordTok{data.frame}\NormalTok{(}\DataTypeTok{group=}\StringTok{"Infant"}\NormalTok{,}\DataTypeTok{value=}\NormalTok{data}\OperatorTok{$}\NormalTok{Infant)}
\NormalTok{b =}\StringTok{ }\KeywordTok{data.frame}\NormalTok{(}\DataTypeTok{group=}\StringTok{"Postneonatal"}\NormalTok{, }\DataTypeTok{value=}\NormalTok{data}\OperatorTok{$}\NormalTok{Postneonatal)}
\NormalTok{c =}\StringTok{ }\KeywordTok{data.frame}\NormalTok{(}\DataTypeTok{group=}\StringTok{"Neonatal Under 28 days"}\NormalTok{, }\DataTypeTok{value=}\NormalTok{data}\OperatorTok{$}\StringTok{`}\DataTypeTok{Neonatal1 Under 28 days}\StringTok{`}\NormalTok{)}
\NormalTok{d =}\StringTok{ }\KeywordTok{data.frame}\NormalTok{(}\DataTypeTok{group=}\StringTok{"Neonatal Under 7 days"}\NormalTok{, }\DataTypeTok{value=}\NormalTok{data}\OperatorTok{$}\StringTok{`}\DataTypeTok{Neonatal1 Under 7 days}\StringTok{`}\NormalTok{)}
\NormalTok{plotdata =}\StringTok{ }\KeywordTok{rbind}\NormalTok{(d,c,b,a)}
\KeywordTok{ggplot}\NormalTok{(}\DataTypeTok{data =}\NormalTok{ plotdata, }\DataTypeTok{mapping =} \KeywordTok{aes}\NormalTok{(}\DataTypeTok{x=}\NormalTok{group,}\DataTypeTok{y=}\NormalTok{value))}\OperatorTok{+}\KeywordTok{geom_boxplot}\NormalTok{()}\OperatorTok{+}\KeywordTok{coord_flip}\NormalTok{()}\OperatorTok{+}
\StringTok{  }\KeywordTok{labs}\NormalTok{(}\DataTypeTok{title=}\StringTok{"Infant mortality rate of different age groups"}\NormalTok{,}\DataTypeTok{y=}\StringTok{"Deaths per 1,000 live births"}\NormalTok{, }\DataTypeTok{x =} \StringTok{""}\NormalTok{)}
\end{Highlighting}
\end{Shaded}

\includegraphics{Assignment-DataAnalysis_files/figure-latex/unnamed-chunk-2-1.pdf}

\begin{Shaded}
\begin{Highlighting}[]
\CommentTok{# Or use following boxplot command:}
\CommentTok{#boxplot(data$Infant, data$`Neonatal1 Under 28 days`, data$`Neonatal1 Under 7 days`, data$Postneonatal, names = c("Infant", "Neonatal1 Under 28 days","Neonatal1 Under 7 days","Postneonatal"),horizontal=TRUE,las=2, ylab = "Deaths per 1,000 live births")}
\end{Highlighting}
\end{Shaded}

From the boxplot of infant mortality rate above, we could see that the
death rate of postneonatals has smaller value and narrower distribution
than that of neonatals.

\hypertarget{difference-in-mortality-rate-due-to-race}{%
\subsubsection{2.2. Difference in Mortality rate due to
race}\label{difference-in-mortality-rate-due-to-race}}

\begin{Shaded}
\begin{Highlighting}[]
\KeywordTok{ggplot}\NormalTok{(}\DataTypeTok{data =}\NormalTok{ data, }\DataTypeTok{mapping =} \KeywordTok{aes}\NormalTok{(}\DataTypeTok{x =}\NormalTok{ data}\OperatorTok{$}\NormalTok{race, }\DataTypeTok{y=}\NormalTok{data}\OperatorTok{$}\NormalTok{Infant))}\OperatorTok{+}\KeywordTok{geom_boxplot}\NormalTok{()}\OperatorTok{+}\KeywordTok{coord_flip}\NormalTok{()}\OperatorTok{+}
\StringTok{  }\KeywordTok{labs}\NormalTok{(}\DataTypeTok{title=}\StringTok{"Infant mortality rate of different races"}\NormalTok{,}\DataTypeTok{y=}\StringTok{"Deaths per 1,000 live births"}\NormalTok{, }\DataTypeTok{x =} \StringTok{""}\NormalTok{)}
\end{Highlighting}
\end{Shaded}

\includegraphics{Assignment-DataAnalysis_files/figure-latex/unnamed-chunk-3-1.pdf}
From the boxplot of infant mortality rate of different races, we could
find that: 1) The white infants have lower death rate than black or
African American infants. 2) The infants born by white mothers have
lower death rate than that by black or African American mothers. 3) The
death rate of black or African American infants is the highest among the
five race groups and has the widest expansion.

It is reasonable to suggest there might be some correlation between the
race of mother and that of child. The large difference between the death
rate of black or African American infants and that of infants with black
or African American mothers seems a little weird. This may due to the
fact that data of different race groups are collected in different
years.

\hypertarget{difference-in-mortality-rate-due-to-time}{%
\subsubsection{2.3. Difference in Mortality rate due to
time}\label{difference-in-mortality-rate-due-to-time}}

\begin{Shaded}
\begin{Highlighting}[]
\KeywordTok{ggplot}\NormalTok{(data, }\KeywordTok{aes}\NormalTok{(}\DataTypeTok{x=}\NormalTok{Year, }\DataTypeTok{y=}\NormalTok{Infant, }\DataTypeTok{shape=}\NormalTok{race, }\DataTypeTok{color=}\NormalTok{race)) }\OperatorTok{+}
\StringTok{  }\KeywordTok{geom_point}\NormalTok{()}\OperatorTok{+}
\StringTok{  }\KeywordTok{labs}\NormalTok{(}\DataTypeTok{title=}\StringTok{"Infant mortality rate trend"}\NormalTok{,}\DataTypeTok{y=}\StringTok{"Deaths per 1,000 live births"}\NormalTok{)}
\end{Highlighting}
\end{Shaded}

\includegraphics{Assignment-DataAnalysis_files/figure-latex/unnamed-chunk-4-1.pdf}
From the scatter plot above, we could see that the infant mortality rate
decreases with time and the decrease trend becomes flatter with time.
And the trend is similar for different races while the infant mortality
rate of white mother (or child) is less than that of black or African
American mother (or child).

\hypertarget{infant-mortality-rate-correlation-between-different-races}{%
\subsubsection{2.4. Infant mortality rate correlation between different
races}\label{infant-mortality-rate-correlation-between-different-races}}

\begin{Shaded}
\begin{Highlighting}[]
\KeywordTok{cor.test}\NormalTok{(data[data}\OperatorTok{$}\NormalTok{race}\OperatorTok{==}\StringTok{'Race of mother White'}\NormalTok{,]}\OperatorTok{$}\NormalTok{Infant, data[data}\OperatorTok{$}\NormalTok{race}\OperatorTok{==}\StringTok{'Race of mother Black or African American'}\NormalTok{,]}\OperatorTok{$}\NormalTok{Infant)}
\end{Highlighting}
\end{Shaded}

\begin{verbatim}
## 
##  Pearson's product-moment correlation
## 
## data:  data[data$race == "Race of mother White", ]$Infant and data[data$race == "Race of mother Black or African American", data[data$race == "Race of mother White", ]$Infant and     ]$Infant
## t = 24.7592, df = 35, p-value < 2.2e-16
## alternative hypothesis: true correlation is not equal to 0
## 95 percent confidence interval:
##  0.9470662 0.9859267
## sample estimates:
##       cor 
## 0.9726198
\end{verbatim}

\begin{Shaded}
\begin{Highlighting}[]
\KeywordTok{cor.test}\NormalTok{(data[(data}\OperatorTok{$}\NormalTok{race}\OperatorTok{==}\StringTok{'All races'}\NormalTok{) }\OperatorTok{&}\StringTok{ }\NormalTok{(data}\OperatorTok{$}\NormalTok{Year}\OperatorTok{>}\DecValTok{1979}\NormalTok{) ,]}\OperatorTok{$}\NormalTok{Infant, data[data}\OperatorTok{$}\NormalTok{race}\OperatorTok{==}\StringTok{'Race of mother Black or African American'}\NormalTok{,]}\OperatorTok{$}\NormalTok{Infant)}
\end{Highlighting}
\end{Shaded}

\begin{verbatim}
## 
##  Pearson's product-moment correlation
## 
## data:  data[(data$race == "All races") & (data$Year > 1979), ]$Infant and data[data$race == "Race of mother Black or African American", data[(data$race == "All races") & (data$Year > 1979), ]$Infant and     ]$Infant
## t = 33.0071, df = 35, p-value < 2.2e-16
## alternative hypothesis: true correlation is not equal to 0
## 95 percent confidence interval:
##  0.9695058 0.9919607
## sample estimates:
##       cor 
## 0.9843141
\end{verbatim}

\begin{Shaded}
\begin{Highlighting}[]
\KeywordTok{cor.test}\NormalTok{(data[data}\OperatorTok{$}\NormalTok{race}\OperatorTok{==}\StringTok{'Race of child White'}\NormalTok{,]}\OperatorTok{$}\NormalTok{Infant, data[data}\OperatorTok{$}\NormalTok{race}\OperatorTok{==}\StringTok{'Race of child Black or African American'}\NormalTok{,]}\OperatorTok{$}\NormalTok{Infant)}
\end{Highlighting}
\end{Shaded}

\begin{verbatim}
## 
##  Pearson's product-moment correlation
## 
## data:  data[data$race == "Race of child White", ]$Infant and data[data$race == "Race of child Black or African American", data[data$race == "Race of child White", ]$Infant and     ]$Infant
## t = 5.5312, df = 2, p-value = 0.03117
## alternative hypothesis: true correlation is not equal to 0
## 95 percent confidence interval:
##  0.1124995 0.9993720
## sample estimates:
##       cor 
## 0.9688344
\end{verbatim}

In this part, we examine the correlation between infant mortality rate
of different races: race of mother white vs.~race of mother black or
African American, race of mother white vs.~all races, race of child
white vs.~race of child black or African American. From the result
above, we could see that all three correlations are significant and
high, almost linear. This correlation result agrees with the trend plot
we have in Part 2.3.


\end{document}
