\RequirePackage{lineno}

\documentclass[onecolumn,aps,prd,groupedaddress,nofootinbib,notitlepage,12pt]{revtex4-1}
%\documentclass[onecolumn,aps,prd,groupedaddress,nofootinbib,notitlepage,12pt]{revtex4}
%\documentclass[preprint,onecolumn,aps,prd,groupedaddress,nofootinbib,notitlepage]{revtex4}
\usepackage{graphicx}
%\graphicspath{{figures/}} \DeclareGraphicsExtensions{.eps}
\usepackage{amsmath}
\usepackage{bm}
\usepackage{slashed}
%\usepackage[hypertex]{hyperref}
\usepackage{xcolor}
\usepackage{amsfonts}
\usepackage{soul}
\usepackage[normalem]{ulem}

%\allowdisplaybreaks allows page breaks inside environment such as align.
\allowdisplaybreaks

%%%%%%%%%%%%%%%%%%%%%%%%%%%%%%%%%%%
%\linenumbers
%%%%%%%%%%%%%%%%%%%%%%%%%%%%%%%%%%%





%%%%%%%%%%%%%%%%%%%%%%%%%%%%%%%%%%
%%    											  %%
%%                           		titles 							  %%
%%    											  %%
%%%%%%%%%%%%%%%%%%%%%%%%%%%%%%%%%%

\begin{document}

\title{Literature Review -- Students' View of Flipped Classroom}

\author{Jing Qian (jq2282)}
%\email{qianjing8758@gmail.com}
%\affiliation{Department of Physics, The Ohio State University, Columbus, Ohio 43210, USA}

\date{\today}

\maketitle
%%%%%%%%%%%%%%%%%%%%%%%%%%%%%%%%%
\section{Instructions}
1. General research 

2. explain logial basis for your experiment

3. closing: state specific goal. variables.(indicate expected results)


\section{Abstract}
1. Definition and history of flipped classroom.
#around the world, cover different majors->chemistry, education, CS, medical...

2. People's research history of flipped classroom.

3. Student's view of flipped classroom.

%%%%%%%%%%%%%%%%%%%%%%%%%%%%%%%%%
\section{Part1}

"Flipped classrooms refer to the practice of assigning lectures outside of class and
devoting class time to a variety of learning activities." (https://pdfs.semanticscholar.org/3034/850a35f8aa62d5218de341872fb3edf9c92b.pdf?_ga=2.156728370.889857229.1550502378-336897620.1550502378)

1. date back to 2006

student centered

"Bergmann and Sams often
repeat, it is a combination of different effective teaching strategies, employed
in different ways by different teachers. This makes flipped learning classes hard
to compare and to make generalisations about them. "


%%%%%%%%%%%%%%%%%%%%%%%%%%%%%%%%%
\section{Part2}

propose six testable propositions about the flipped classroom\cite{Abeysekera:2015}

%%%%%%%%%%%%%%%%%%%%%%%%%%%%%%%%%
\section{Part3}


%%%%%%%%%%%%%%%%%%%%%%%%%%%%%%%%%
\section{Flipped Courses}
review :
https://pdfs.semanticscholar.org/5710/14a2d984dbfeea58f68b28860a07e59047dc.pdf?_ga=2.119126144.889857229.1550502378-336897620.1550502378
\\\\
2. Do students like the flipped classroom? An investigation of student reaction to a flipped undergraduate IT course (2014)

https://www.semanticscholar.org/paper/Do-students-like-the-flipped-classroom-An-of-to-a-Elliott/7e877481ac9ae33d25b404796d86258e61528f17

说的含糊,combines successful techniques for distance education with constructivist learning theory in the classroom. 但说是significant improvement in learning outcomes.
\\\\
2.*. similarly, china, zhejiang university, 15 undergraduate education major students,  students satisfied with many aspects, but not interaction with others. https://online-journals.org/index.php/i-jet/article/view/4708

\\\\
3. The flipped classroom: for active, effective and increased learning – especially for low achievers (2016)

http://educationaltechnologyjournal.springeropen.com/articles/10.1186/s41239-016-0032-z

Questionnaire, majority positive.  Low achievers significantly reported more positively.
\\\\
5.  Performance and perception in the flipped classroom (2016)

https://link.springer.com/article/10.1007\%2Fs10639-015-9393-5

University of the West Indies, undergraduate, two successive years -- 1 traditional vs. 1 flip.
Evaluate learning experience -> exam performance & student perception.
Grades vs. student perception.
qualitative -> slight improvement on perception, quantitative -> no significant change in grade, fewer highest remarks.
\\\\
6. Implementing the flipped classroom: an exploration of study behaviour and student performance (2016)

https://pdfs.semanticscholar.org/4005/8eb035a8821edd7392a12ce6ee576a74d1e0.pdf?_ga=2.89775250.889857229.1550502378-336897620.1550502378

learning behavior-> bi-weekly diaries. 
study behavior similar with regular, but behavior not strongly related to performance.
Some resistance to change behavior.


\\\\
4. The Flipped-Classroom Approach: The Answer to Future Learning? (2014)

https://www.semanticscholar.org/paper/The-Flipped-Classroom-Approach\%3A-The-Answer-to-Kurtz-Tsimerman/151f1ad5f1465d5c0cb53a939366df5e5cc16990

Isreal, quesionnaire,  business undergraduate.
Positive attitude.
Better: in class, non-worker, female, older, have classmate nearby
\\\\
1. Student views on the use of flipped learning in higher education: A pilot study (2019)

https://link.springer.com/article/10.1007\%2Fs10639-019-09881-8

Pilot study. Questionnaire from students.
Pros: flexibility, responsible. 
Cons: lack immediate feedback \& poor internet accessibility.
\\\\
7. Faculty Support for Effective Flipped Classrooms in Higher Education (2017)

https://www.semanticscholar.org/paper/Faculty-Support-for-Effective-Flipped-Classrooms-in-Iwasaki/2a4aa7bbeb9c7ce3db791feb7cc482e707d5a9b6

Japan, questionnaire survey\&interview of  faculty
\\\\
8. 

https://www.sciencedirect.com/science/article/pii/S0360131517300039?via\%3Dihub

90 second year undergraduate, invertebrates corses, Vietnam.
performance superior in flipped classroom. 
positive effect on self-efficacy beliefs and intrinsic motivation, but not on perceived flexibility 
variables: learning performance, self-efficacy, intrinsic motivation, perceived flexibility.




%%%%%%%%%%%%%%%%%%%%%%%%%%%%%%%%%
%                               %
%          References           %
%  						  	    %
%%%%%%%%%%%%%%%%%%%%%%%%%%%%%%%%%


\begin{thebibliography}{99}
\bibitem{Abeysekera:2015}
Abeysekera, Lakmal and Dawson, Phillip,
Higher Education Research \& Development {\bf 34}, 1 (2015)
doi: 10.1080/07294360.2014.934336

%\bibliography{./references/mybibliography}
\end{thebibliography}

%======================= The End =================================
\end{document}

