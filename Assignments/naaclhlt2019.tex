%
% File naacl2019.tex
%
%% Based on the style files for ACL 2018 and NAACL 2018, which were
%% Based on the style files for ACL-2015, with some improvements
%%  taken from the NAACL-2016 style
%% Based on the style files for ACL-2014, which were, in turn,
%% based on ACL-2013, ACL-2012, ACL-2011, ACL-2010, ACL-IJCNLP-2009,
%% EACL-2009, IJCNLP-2008...
%% Based on the style files for EACL 2006 by 
%%e.agirre@ehu.es or Sergi.Balari@uab.es
%% and that of ACL 08 by Joakim Nivre and Noah Smith

\documentclass[11pt,a4paper]{article}
\usepackage[hyperref]{naaclhlt2019}
\usepackage{times}
\usepackage{latexsym}

\usepackage{url}

%\aclfinalcopy % Uncomment this line for the final submission
%\def\aclpaperid{***} %  Enter the acl Paper ID here

%\setlength\titlebox{5cm}
% You can expand the titlebox if you need extra space
% to show all the authors. Please do not make the titlebox
% smaller than 5cm (the original size); we will check this
% in the camera-ready version and ask you to change it back.

\newcommand\BibTeX{B{\sc ib}\TeX}

\title{Literature Review -- Does the flipped classroom increase the learning efficiency?}

\author{Jing Qian (jq2282)
  }

\date{}

\begin{document}
\maketitle

%\begin{abstract}

%1. General research 
%
%2. explain logial basis for your experiment
%
%3. closing: state specific goal. variables.(indicate expected results)
%\end{abstract}

%\section{Part1}
The flipped classroom, a new strategy of teaching, has gained growing research interest especially in the higher education community.
As a combination of different effective teaching strategies and employed
in different ways by different teachers, the main idea of flipped classroom is simple: assign lectures outside of class and
devote class time to a variety of learning activities\cite{DeLozier:2016}.
Contrary to the traditional lecturer-centered teaching, the flipped classroom is more flexible and student-centered. 
As a new pedagogy which dates back to 2006, thanks to technical advances combined with increased student comfort in an online learning environment, flipped classroom has become very prevalent in different subjects around the whole world.
Supporters argued that as well as utilizing the educational resource and benefiting from adopting new digital technologies, this pedagogy method could improve student learning\cite{Abeysekera:2015}.

Since 2012, there have been booming research papers about flipped classroom method, including faculty's view, discussion of how to implement flipped classrooms and students' perceptions.
Primarily, studies investigating the impact of the flipped classroom phenomenon  looked at students' perceptions and many of them reported very positive results.
\cite{Kurtz:2014} reported a study examing business major undergraduates' experience of flipped classroom by questionnaire. The students appreciated watching videos between lessons for interest enhancement and better understanding of learning material. They used multivariate analysis and point out that non-working students, female students and older students are more positive. 
In the questionnaire study of 240 undergraduates, \cite{Nouri:2016} argued that students generally appreciate flipped classroom due to the flexibility to use video material in their own pace, especially for the students with lower average grades.
In the pilot study from \cite{Gündüz:2019}, questionnaire from students showed that students experience greater instructional flexibility and more responsibilities for their own learning. There are some disappointment due to lacking immediate feedback and poor Internet accessibility.

While there is a generally positive perception of the flipped classroom, fewer studies focused on the outcomes and there is "not strong enough evidence to support the claim that student learning was enhanced by this format"\cite{DeLozier:2016}.
Students' perceptions of learning are not tantamount to objective measures of learning performance. Accordingly, any evaluation and review of flipped classrooms should ideally be guided by objective measures of learning.
\cite{Blair:2016} compared the undergraduate performance between traditional and flipped classrooms by analysis of course grades and reported that no significant change  of average cohort exam performance was found, fewer students in the flipped classroom achieved marks at the highest level.
In the study of 90 second year undergraduates, \cite{Thai:2017} suggested that students in the flipped classroom has superior learning performance compared to traditional learning and several other learning methods. 

Other than the students' performance, student engagement, which is commonly defined as the time and energy students invest in educationally purposeful activities is also one important factor in the education studies.
After examining these studies of flipped classroom pedagogy, I have  found few papers discussing the workload of flipped classroom and how it compares with traditional teaching method.
Even if students have better performance in flipped classroom, if they spend much more time than in the traditional classroom, the  flipped classroom may not be an effective way of teaching.
Also, the workload itself may contribute to the students' perceptions of certain teaching method.

There are some papers mentioning that students spent more time in the pre-class preparing part in the flipped classroom\cite{Blair:2016}, but I find no study examining the total time, which is the summation of pre-class, in-class and after-class work time, and compare it with other pedagogues.behavior
One interesting paper is about study behavior and student performance in the flipped classroom.
\cite{Boevé:2016} explored the students' study behavior by means of bi-weekly diaries.
They argued that the study hour and study behavior in the flipped course "did not appear to be very different from that of students in a regular course".
However, in their study, the students who took flipped classroom and those took traditional classroom were from different major: the students in the flipped course were enrolled in the pedagogical science major, and the students in the regular course were enrolled in the psychology major. 
This different background may have some effects on the observed results.
Also, they didn't mention the details of their study time: How much time did students spend on pre-class, in-class and after-class study? 
Did the study time include each sections?
It may help if we look deeply into that.

In our study, we would like to shed some light on the following question:
Does the flipped classroom promote the learning efficiency of students?
Here we quantify the learning efficiency with two variables: learning performance (LP) which shows the amount that students learn from the course and workload (W) which shows how much time students spend on the specific course.
Learning performance (LP) could be evaluated by students' exam performance (in-class quiz and final exams), and the workload (W) could be evaluated by their hours of study on the specific course. 
We investigate two sections of the same course, one is traditional classroom while the other is flipped classroom.
If flipped classroom method leads to higher learning performance and lower workload compared to the traditional one, we would say that the flipped classroom method increase the learning efficiency and vice versa.

We argue that from our study, although some students may have better flipped learning experience, due to similar learning performance and heavier workload in the flipped classroom than the traditional classroom, the flipped classroom method leads to lower learning efficiency than the traditional way.






\bibliography{naaclhlt2019}
\bibliographystyle{acl_natbib}


\end{document}
